\chapter{例子}
\section{图片}
图片统一放在image路径下,在插入图片的时候写图片相对路径。

多个子图\ref{fig:logos:a}\ref{fig:logos:b}\ref{fig:logos:c}

\begin{figure}[hp]
\centering
\subfigure[]
{\label{fig:logos:a}
\includegraphics[angle=0, width=0.3\textwidth]{./image/cs-logo.png}}
\subfigure[]
{\label{fig:logos:b}
\includegraphics[angle=0, width=0.3\textwidth]{./image/cs-logo.png}}
\subfigure[]
{\label{fig:logos:c}
\includegraphics[angle=0, width=0.3\textwidth]{./image/cs-logo.png}}
\caption{多个计算机学院logo}
\label{fig:logos}
\end{figure}

单独一图\ref{fig:logo}
\begin{figure}[hp]
\centering
\includegraphics[angle=0, width=0.7\textwidth]{./image/cs-logo.png}
\caption{一个计算机学院logo}
\label{fig:logo}
\end{figure}


\section{公式}
行内公式$\mathbf{x}(s)=[x(s),y(s)],s\in[0,1]$,带编号的公式
\begin{equation} \label{eq:energy}
E=\int_{0}^{1}\frac{1}{2}(\alpha|\mathbf{x}'(s)|^{2}+\beta|\mathbf{x}''(s)|^{2})+E_{ext}(\mathbf{x}(s)) \ud s
\end{equation}

无编号的公式
\begin{displaymath}
u_{t}=\frac{1}{\Delta t}(u_{i,j}^{n+1}-u_{i,j}^{n})
\end{displaymath}

\section{参考文献}
请看论文\cite{fem}.
